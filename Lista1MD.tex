\documentclass[12pt]{article}
\usepackage[margin=1in]{geometry}
\usepackage[all]{xy}


\usepackage{amsmath,amsthm,amssymb,color,latexsym}
\usepackage{geometry}        
\geometry{letterpaper}    
\usepackage{graphicx}

\newtheorem{problem}{}

\newenvironment{solution}[1][\it{\textbf{Solução}}]{\textbf{#1. } }


\begin{document}
\noindent Matemática discreta Turma CC 2023.2\hfill Primeira lista da 1a avaliação \\
Tomé da Costa Lima

\hrulefill


\begin{problem}
Prove pelo método da contrapositiva que, se b é um número ímpar, então a equação $x^2 - x - b^2 = 0$ não tem soluções inteiras
\end{problem}
\begin{solution}
	Assuma x, b inteiros tais que $x^2 - x - b^2 = 0$. 
 \begin{align*}
      x^2 - x - b^2 &= 0 \\
        x^2 - x &= b^2    
 \end{align*}   
 Podemos dividir em dois casos:
 
 \textbf{1}: x é par, existindo n inteiro tal que $x = 2n$:
 \begin{align*}
     (2n)^2 - 2n = b^2 \\
     4n^2 - 2n = b^2 \\
     2 (2n^2 - n) = b^2
 \end{align*}

sendo $b^2$ par,  b também é \textbf{par}.

\textbf{2.} x é ímpar, existindo m inteiro tal que x = 2m + 1.
\begin{align*}
    (2m + 1)^2 - (2m + 1) = b^2 \\
    4m^2 2 + 4m + 1 - 2m - 1 = b^2 \\
    2(m^2 + 2m) = b^2
\end{align*}
Sendo $b^2$ par, b também é \textbf{par}.

Assim, vê-se que, para que a equação $x^2 - x - b^2 = 0$ possa ter uma solução inteira, b não pode ser ímpar.
\end{solution}

\begin{problem}
Prove as seguintes proposições:

a) A multiplicação entre número racional e um irracional é irracional. 

b) entre dois números racionais, há um irracional 
\end{problem}
\begin{solution}
	
 \textbf{a)} Assuma $q \in \mathbb{Q}$ e ${i \in \mathbb{I}}$ tais que $qi = \frac{a}{b}$ sendo a e b inteiros coprimos.
 \begin{align*}
     qi  = \frac{a}{b}
 \end{align*}
 sendo q racional, existem inteiros coprimos r e s tais que $q = \frac{r}{s}$
 \begin{align*}
     \frac{r}{s}i = \frac{a}{b} \\
     i = \frac{as}{br}
 \end{align*}
Vê-se que i é racioal e irracional, o que é uma contradição. Portanto, $qi \in \mathbb{I}$.

\newpage

\textbf{b)} Assuma p e q racionais. Sendo p racional, existem inteiros coprimos a e b tais que $p = \frac{a}{b}$. Analogamente, existem c e d inteiros coprimos tais que $q = \frac{c}{d}$.

Sendo $ p < q$, podemos dividir em dois casos:

\textbf{p $>$ 0}: sendo p positivo, qualquer número i entre p e q pode ser escrito na forma $ i = p + x \mid 0 < x < (q - p)$
\begin{align*}
    i = \frac{a}{b} + x \mid x < (\frac{c}{d} - \frac{a}{b}) \\
   x < \frac{bc - ad}{bd} \\
\end{align*}
Sendo $0 < x< \frac{bc - ad}{bd}$, um valor válido para x seria $x = \frac{bc - ad}{bd\sqrt{2}}$, uma vez que \newline$ 0 < \frac{bc - ad}{bd\sqrt{2}} = \frac{bc - ad}{bd} \times \frac{1}{\sqrt{2}} < \frac{bc - ad}{bd}$. Portanto:
\begin{align*}
    \frac{a}{b} < (\frac{a}{b} + \frac{bc - ad}{\sqrt{2}bd})\ < \frac{c}{d} \\
    \frac{a}{b} < \frac{bda\sqrt{2} + b^2c - adb}{b^2d\sqrt{2}}
\end{align*}

\end{solution}
 

\begin{problem}
Write here the text of the second homework problem.
\end{problem}
\begin{solution}
	Write here the solution of the second homework problem.
\end{solution}
 

\begin{problem}
Write here the text of the second homework problem.
\end{problem}
\begin{solution}
	Write here the solution of the second homework problem.
\end{solution}
 

%%%%%%%%%%%%%%%%%%%%%%%%%%%%%%%%%%%%%%%%%%%%%%%%%%%%%%%%
%%%%%Continue with this pattern if there are more%%%%%%%
%%%%%%%%%%%%%%%%%homework problems%%%%%%%%%%%%%%%%%%%%%%
%%%%%%%%%%%%%%%%%%%%%%%%%%%%%%%%%%%%%%%%%%%%%%%%%%%%%%%%
 
\end{document}
